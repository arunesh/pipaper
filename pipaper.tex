
\documentclass[journal]{IEEEtran}
% Some very useful LaTeX packages include:
% (uncomment the ones you want to load)


% *** MISC UTILITY PACKAGES ***
%
%\usepackage{ifpdf}
% Heiko Oberdiek's ifpdf.sty is very useful if you need conditional
% compilation based on whether the output is pdf or dvi.
% usage:
% \ifpdf
%   % pdf code
% \else
%   % dvi code
% \fi
% The latest version of ifpdf.sty can be obtained from:
% http://www.ctan.org/pkg/ifpdf
% Also, note that IEEEtran.cls V1.7 and later provides a builtin
% \ifCLASSINFOpdf conditional that works the same way.
% When switching from latex to pdflatex and vice-versa, the compiler may
% have to be run twice to clear warning/error messages.



% *** CITATION PACKAGES ***
%
\usepackage{cite}

% *** GRAPHICS RELATED PACKAGES ***
%
\ifCLASSINFOpdf
  % \usepackage[pdftex]{graphicx}
  % declare the path(s) where your graphic files are
  % \graphicspath{{../pdf/}{../jpeg/}}
  % and their extensions so you won't have to specify these with
  % every instance of \includegraphics
  % \DeclareGraphicsExtensions{.pdf,.jpeg,.png}
\else
  % or other class option (dvipsone, dvipdf, if not using dvips). graphicx
  % will default to the driver specified in the system graphics.cfg if no
  % driver is specified.
  % \usepackage[dvips]{graphicx}
  % declare the path(s) where your graphic files are
  % \graphicspath{{../eps/}}
  % and their extensions so you won't have to specify these with
  % every instance of \includegraphics
  % \DeclareGraphicsExtensions{.eps}
\fi
% graphicx was written by David Carlisle and Sebastian Rahtz. It is
% required if you want graphics, photos, etc. graphicx.sty is already
% installed on most LaTeX systems. The latest version and documentation
% can be obtained at: 
% http://www.ctan.org/pkg/graphicx
% Another good source of documentation is "Using Imported Graphics in
% LaTeX2e" by Keith Reckdahl which can be found at:
% http://www.ctan.org/pkg/epslatex
%
% latex, and pdflatex in dvi mode, support graphics in encapsulated
% postscript (.eps) format. pdflatex in pdf mode supports graphics
% in .pdf, .jpeg, .png and .mps (metapost) formats. Users should ensure
% that all non-photo figures use a vector format (.eps, .pdf, .mps) and
% not a bitmapped formats (.jpeg, .png). The IEEE frowns on bitmapped formats
% which can result in "jaggedy"/blurry rendering of lines and letters as
% well as large increases in file sizes.
%
% You can find documentation about the pdfTeX application at:
% http://www.tug.org/applications/pdftex





% *** MATH PACKAGES ***
%
\usepackage{amsmath}
% A popular package from the American Mathematical Society that provides
% many useful and powerful commands for dealing with mathematics.
%
% Note that the amsmath package sets \interdisplaylinepenalty to 10000
% thus preventing page breaks from occurring within multiline equations. Use:
%\interdisplaylinepenalty=2500
% after loading amsmath to restore such page breaks as IEEEtran.cls normally
% does. amsmath.sty is already installed on most LaTeX systems. The latest
% version and documentation can be obtained at:
% http://www.ctan.org/pkg/amsmath





% *** SPECIALIZED LIST PACKAGES ***
%
%\usepackage{algorithmic}
% algorithmic.sty was written by Peter Williams and Rogerio Brito.
% This package provides an algorithmic environment fo describing algorithms.
% You can use the algorithmic environment in-text or within a figure
% environment to provide for a floating algorithm. Do NOT use the algorithm
% floating environment provided by algorithm.sty (by the same authors) or
% algorithm2e.sty (by Christophe Fiorio) as the IEEE does not use dedicated
% algorithm float types and packages that provide these will not provide
% correct IEEE style captions. The latest version and documentation of
% algorithmic.sty can be obtained at:
% http://www.ctan.org/pkg/algorithms
% Also of interest may be the (relatively newer and more customizable)
% algorithmicx.sty package by Szasz Janos:
% http://www.ctan.org/pkg/algorithmicx




% *** ALIGNMENT PACKAGES ***
%
%\usepackage{array}
% Frank Mittelbach's and David Carlisle's array.sty patches and improves
% the standard LaTeX2e array and tabular environments to provide better
% appearance and additional user controls. As the default LaTeX2e table
% generation code is lacking to the point of almost being broken with
% respect to the quality of the end results, all users are strongly
% advised to use an enhanced (at the very least that provided by array.sty)
% set of table tools. array.sty is already installed on most systems. The
% latest version and documentation can be obtained at:
% http://www.ctan.org/pkg/array


% IEEEtran contains the IEEEeqnarray family of commands that can be used to
% generate multiline equations as well as matrices, tables, etc., of high
% quality.




% *** SUBFIGURE PACKAGES ***
%\ifCLASSOPTIONcompsoc
%  \usepackage[caption=false,font=normalsize,labelfont=sf,textfont=sf]{subfig}
%\else
%  \usepackage[caption=false,font=footnotesize]{subfig}
%\fi
% subfig.sty, written by Steven Douglas Cochran, is the modern replacement
% for subfigure.sty, the latter of which is no longer maintained and is
% incompatible with some LaTeX packages including fixltx2e. However,
% subfig.sty requires and automatically loads Axel Sommerfeldt's caption.sty
% which will override IEEEtran.cls' handling of captions and this will result
% in non-IEEE style figure/table captions. To prevent this problem, be sure
% and invoke subfig.sty's "caption=false" package option (available since
% subfig.sty version 1.3, 2005/06/28) as this is will preserve IEEEtran.cls
% handling of captions.
% Note that the Computer Society format requires a larger sans serif font
% than the serif footnote size font used in traditional IEEE formatting
% and thus the need to invoke different subfig.sty package options depending
% on whether compsoc mode has been enabled.
%
% The latest version and documentation of subfig.sty can be obtained at:
% http://www.ctan.org/pkg/subfig




% *** FLOAT PACKAGES ***
%
%\usepackage{fixltx2e}
% fixltx2e, the successor to the earlier fix2col.sty, was written by
% Frank Mittelbach and David Carlisle. This package corrects a few problems
% in the LaTeX2e kernel, the most notable of which is that in current
% LaTeX2e releases, the ordering of single and double column floats is not
% guaranteed to be preserved. Thus, an unpatched LaTeX2e can allow a
% single column figure to be placed prior to an earlier double column
% figure.
% Be aware that LaTeX2e kernels dated 2015 and later have fixltx2e.sty's
% corrections already built into the system in which case a warning will
% be issued if an attempt is made to load fixltx2e.sty as it is no longer
% needed.
% The latest version and documentation can be found at:
% http://www.ctan.org/pkg/fixltx2e


%\usepackage{stfloats}
% stfloats.sty was written by Sigitas Tolusis. This package gives LaTeX2e
% the ability to do double column floats at the bottom of the page as well
% as the top. (e.g., "\begin{figure*}[!b]" is not normally possible in
% LaTeX2e). It also provides a command:
%\fnbelowfloat
% to enable the placement of footnotes below bottom floats (the standard
% LaTeX2e kernel puts them above bottom floats). This is an invasive package
% which rewrites many portions of the LaTeX2e float routines. It may not work
% with other packages that modify the LaTeX2e float routines. The latest
% version and documentation can be obtained at:
% http://www.ctan.org/pkg/stfloats
% Do not use the stfloats baselinefloat ability as the IEEE does not allow
% \baselineskip to stretch. Authors submitting work to the IEEE should note
% that the IEEE rarely uses double column equations and that authors should try
% to avoid such use. Do not be tempted to use the cuted.sty or midfloat.sty
% packages (also by Sigitas Tolusis) as the IEEE does not format its papers in
% such ways.
% Do not attempt to use stfloats with fixltx2e as they are incompatible.
% Instead, use Morten Hogholm'a dblfloatfix which combines the features
% of both fixltx2e and stfloats:
%
% \usepackage{dblfloatfix}
% The latest version can be found at:
% http://www.ctan.org/pkg/dblfloatfix




%\ifCLASSOPTIONcaptionsoff
%  \usepackage[nomarkers]{endfloat}
% \let\MYoriglatexcaption\caption
% \renewcommand{\caption}[2][\relax]{\MYoriglatexcaption[#2]{#2}}
%\fi
% endfloat.sty was written by James Darrell McCauley, Jeff Goldberg and 
% Axel Sommerfeldt. This package may be useful when used in conjunction with 
% IEEEtran.cls'  captionsoff option. Some IEEE journals/societies require that
% submissions have lists of figures/tables at the end of the paper and that
% figures/tables without any captions are placed on a page by themselves at
% the end of the document. If needed, the draftcls IEEEtran class option or
% \CLASSINPUTbaselinestretch interface can be used to increase the line
% spacing as well. Be sure and use the nomarkers option of endfloat to
% prevent endfloat from "marking" where the figures would have been placed
% in the text. The two hack lines of code above are a slight modification of
% that suggested by in the endfloat docs (section 8.4.1) to ensure that
% the full captions always appear in the list of figures/tables - even if
% the user used the short optional argument of \caption[]{}.
% IEEE papers do not typically make use of \caption[]'s optional argument,
% so this should not be an issue. A similar trick can be used to disable
% captions of packages such as subfig.sty that lack options to turn off
% the subcaptions:
% For subfig.sty:
% \let\MYorigsubfloat\subfloat
% \renewcommand{\subfloat}[2][\relax]{\MYorigsubfloat[]{#2}}
% However, the above trick will not work if both optional arguments of
% the \subfloat command are used. Furthermore, there needs to be a
% description of each subfigure *somewhere* and endfloat does not add
% subfigure captions to its list of figures. Thus, the best approach is to
% avoid the use of subfigure captions (many IEEE journals avoid them anyway)
% and instead reference/explain all the subfigures within the main caption.
% The latest version of endfloat.sty and its documentation can obtained at:
% http://www.ctan.org/pkg/endfloat
%
% The IEEEtran \ifCLASSOPTIONcaptionsoff conditional can also be used
% later in the document, say, to conditionally put the References on a 
% page by themselves.




% *** PDF, URL AND HYPERLINK PACKAGES ***
%
\usepackage{url}
% url.sty was written by Donald Arseneau. It provides better support for
% handling and breaking URLs. url.sty is already installed on most LaTeX
% systems. The latest version and documentation can be obtained at:
% http://www.ctan.org/pkg/url
% Basically, \url{my_url_here}.




% *** Do not adjust lengths that control margins, column widths, etc. ***
% *** Do not use packages that alter fonts (such as pslatex).         ***
% There should be no need to do such things with IEEEtran.cls V1.6 and later.
% (Unless specifically asked to do so by the journal or conference you plan
% to submit to, of course. )


% correct bad hyphenation here
\hyphenation{op-tical net-works semi-conduc-tor}


\begin{document}
%
% paper title
% Titles are generally capitalized except for words such as a, an, and, as,
% at, but, by, for, in, nor, of, on, or, the, to and up, which are usually
% not capitalized unless they are the first or last word of the title.
% Linebreaks \\ can be used within to get better formatting as desired.
% Do not put math or special symbols in the title.
\title{On Designing Performant Peer-Peer Overlay Networks for Web 3.0}
%\title{How fast can the Network get ?.cls\\ for IEEE Journals}
%
%
% author names and IEEE memberships
% note positions of commas and nonbreaking spaces ( ~ ) LaTeX will not break
% a structure at a ~ so this keeps an author's name from being broken across
% two lines.
% use \thanks{} to gain access to the first footnote area
% a separate \thanks must be used for each paragraph as LaTeX2e's \thanks
% was not built to handle multiple paragraphs
%

\author{Arunesh~Mishra~and Adi~Kancherla\\~\IEEEmembership{Picolo Labs}% <-this % stops a space
}% <-this % stops a space

% note the % following the last \IEEEmembership and also \thanks - 
% these prevent an unwanted space from occurring between the last author name
% and the end of the author line. i.e., if you had this:
% 
% \author{....lastname \thanks{...} \thanks{...} }
%                     ^------------^------------^----Do not want these spaces!
%
% a space would be appended to the last name and could cause every name on that
% line to be shifted left slightly. This is one of those "LaTeX things". For
% instance, "\textbf{A} \textbf{B}" will typeset as "A B" not "AB". To get
% "AB" then you have to do: "\textbf{A}\textbf{B}"
% \thanks is no different in this regard, so shield the last } of each \thanks
% that ends a line with a % and do not let a space in before the next \thanks.
% Spaces after \IEEEmembership other than the last one are OK (and needed) as
% you are supposed to have spaces between the names. For what it is worth,
% this is a minor point as most people would not even notice if the said evil
% space somehow managed to creep in.


% The paper headers
\markboth{Extended Abstract submission to Stanford Blockchain Conference, 2019.}%
{Mishra \MakeLowercase{\textit{et al.}}: On Designing Performant Peer-peer Overlay Networks}
% The only time the second header will appear is for the odd numbered pages
% after the title page when using the twoside option.
% 
% *** Note that you probably will NOT want to include the author's ***
% *** name in the headers of peer review papers.                   ***
% You can use \ifCLASSOPTIONpeerreview for conditional compilation here if
% you desire.

% make the title area
\maketitle

% As a general rule, do not put math, special symbols or citations
% in the abstract or keywords.
\begin{abstract}

Many experts believe that decentralized applications will be the core paradigm for Web 3.0. To become real, these applications need to run on a
    fast network. Since most networks provide an open participation model, the system doesn't have control over how the
    network topology grows. In this work, we conduct an empirical analysis of the network layer that supports decentralized applications and blockchains. 
    We present a set of design principles that p2p overlay networks should adhere to in general. The talk will also
    discuss empirical results from data collected on IPFS, Ethereum and a couple of other blockchain projects to compare
    and contrast their design and performance. Our goal is to foster an open discussion so that the network layer can
    scale the applications that run on top of it.

\end{abstract}

% For peer review papers, you can put extra information on the cover
% page as needed:
% \ifCLASSOPTIONpeerreview
% \begin{center} \bfseries EDICS Category: 3-BBND \end{center}
% \fi
%
% For peerreview papers, this IEEEtran command inserts a page break and
% creates the second title. It will be ignored for other modes.
\IEEEpeerreviewmaketitle

\section{Introduction}
\IEEEPARstart{W}{e} are in the midst of building a new Internet. Blockchain networks like Ethereum have popularized the idea of
fully decentralized, owner less applications that are run on an open network of untrusted but incentivized nodes without the
oversight of a central authority. There are many other projects such as Dfinity, ThunderCore, Golem, etc who are
creating the {\em world computer}. This radical computing platform has the potential to solve many of the issues around
data privacy and the paradox of {\it user-trust} in an attention economy \cite{trust_att} incentivized to sell eye-ball
time to advertisers.

In order to realize these benefits, decentralization requires an open participation model, that is, anybody in general
should be allowed to participate on the network. This brings forth new and exciting technical challenges for the Academic
and Research community to tackle. One such challenge is the need for a fast network that connects the
participating nodes, miners or users and their devices. This p2p overlay network \cite{p2p_survey}
is a building block for blockchain protocols or core blockchain operations, such as consensus, sharding, replication,
ledgers etc. All of the communication messages use the overlay network and thus, it becomes important to understand how such
a network should be designed.

This short writeup discusses ongoing work as part of the Picolo Project \footnote{See https://picolo.network/paper for
a detailed discussion of the Picolo network and database layer.}. The goal of the Picolo project is
to build a fast decentralized open data network to support blockchain applications by offloading the storage and
querying of structured data to a separate fast network. As a part of this effort, we are evaluating the network layer designs used by existing
projects in the blockchain space order to understand a few aspects, namely: (i) what are the typical overlay topologies
(and their properties)
for blockchain networks and (ii) what are the routing, lookup, message transfer latencies, costs and inefficiences if
any. This work will help us design a better p2p overlay which is critical to support production quality decentralized
applications if we are to make that future a reality.

The rest of this writeup is organized as follows. We present a brief summary of the related p2p overlay network layers
in Section \ref{sec:rel}. Subsequently in Section \ref{sec:res} we present an overview of our research methodology into
existing p2p overlay designs that support popular blockchains and applications today. Finally, we conclude in Section
\ref{sec:conc}.

% The very first letter is a 2 line initial drop letter followed
% by the rest of the first word in caps.
% 
% form to use if the first word consists of a single letter:
% \IEEEPARstart{A}{demo} file is ....
% 
% form to use if you need the single drop letter followed by
% normal text (unknown if ever used by the IEEE):
% \IEEEPARstart{A}{}demo file is ....
% 
% Some journals put the first two words in caps:
% \IEEEPARstart{T}{his demo} file is ....
% 
% Here we have the typical use of a "T" for an initial drop letter
% and "HIS" in caps to complete the first word.
% You must have at least 2 lines in the paragraph with the drop letter
% (should never be an issue)

%\hfill mds
 
%\hfill August 26, 2015

\section{Related Work}
\label{sec:rel}
There has been extensive research on p2p overlay networks over the last 15 years.  Napster \cite{Napster} was one of the first popular services that provided much of the original inspiration for p2p
systems although its database was centralized.  DNS is an example of a widely deployed distributed and largely
decentralized key-value database that powers every lookup and interaction on the Internet \cite{Mockapetris_1988}. DNS
relies on special root servers to bootstrap the lookup protocol. The Freenet \cite{freenet_thesis, Clarke_2001} and the
Gnutella \cite{Gnutella} p2p systems were popular in the previous decade for file sharing. Both systems were designed
for sharing of large files over a longer duration of time. Content reliabilty including lookup reliability and network
latency goals were necessary in this enviroment. 

The second generation of p2p systems include research driven projects such as Chord \cite{Stoica_2001}, Content
Addressable Network (CAN)
\cite{Ratnasamy_2001}, Pastry \cite{Rowstron_2001}, Tapestry \cite{tapestry2004} and
Kademlia. 

Other notable systems include Viceroy \cite{viceroy} which provides logarithmic hops through nodes with constant degree
routing tables. SkipNet \cite{skipnet} uses a multidimensional skip-list data structure to support overlay routing,
maintaining both a DNS-based namesapce for operational locality and a randomized namespace for network locality. Other
overlay proposals such as Koorde \cite{koorde} and Naor et al \cite{simple_hash} attain lower bounds on local routing
state but oversimplify some of the other features. 
\newline\newline
The third generation of p2p research includes building applications on top of these DHT systems, validating them as
novel infrastructures or tuning them for specific use cases. For example, applications such as PAST \cite{past} and
SCRIBE \cite{scribe} are built on top of Pastry. Decentralized file storage application project OceanStore \cite{oceanstore} was built
on top of Tapestry, while CFS \cite{cfs} was build on top of Chord. FarSite \cite{farsite} uses a conventional
distributed directory service and could be built on top of Pastry. Another example of an overlay network is the Overcast
System \cite{overcast}, which provides reliable single-source multicast streams.

Given the wide diversity of research in the p2p overlay community, it is a bit surprising that the blockchain community
has largely only used one or two variants of the choices available. IPFS uses Kademlia \cite{kademlia} which is
implemented as a part of {\it libp2p} project. Many other network layers have made this choice, while some resort to a
more simplistic gossip-protocol style designs.

\section{Research Methodology}
\label{sec:res}
The goal of our study is to understand the architecture of some of the popular blockchain networks, such as Ethereum
(Whisper protocol), IPFS, Bitcoin alongwith some of the emerging projects in this space. Our approach includes
documenting their overlay network properties according to the features listed the following subsection. Further we have
built a "network crawler" utility, similar to traceroute, that can crawl the layer 4 topology of these networks to
further understand them through certain metrics (discussed below).

Overall, we find that most blockchain networks and decentralized projects pay very little attention to the design
details for the p2p overlay layer. This is expected as the majority of the focus has been on speed, fast consensus and
smart contract execution to name a few challenges. However, a well designed network layer has demonstrated improvements
at the application layer metrics in past work \cite{tapestry2004, farsite, oceanstore, past}. We will present results
from these various networks alongside insights on any bottlenecks and possible improvements both including short-term
fixes and long-term architectural improvements. In the next subsection, we discuss some of the aspects of the routing
layer that are relevant to decentralized applications in general.

\subsection{P2p Overlay Design principles}
Some of the interesting and relevant aspects of a p2p overlay routing system would include:

\begin{itemize}
    \item {\em Location independent routing:} Location independent routing refers to a class of techniques for locating objects based on content rather than
        their location, while attempting to find the shortest path possible to reach such objects. For example, in IPFS
        \cite{ipfs} a hash of the block is used to identify and refer to the content. The hash is also use to locate the
        content using the Kademlia routing protocol. Depending on the application, such as content lookup versus
        synchronizing across consensus shards, this is a

    \item {\em Deterministic node mapping:} It should be possible to route messages or lookup objects and services in the network regardless of where the
        lookup originates. The result of the distributed routing algorithm should be deterministic.

    \item {\em Overlay topology quality metrics:} Overlay metrics such as routing stretch and stress help understand how inefficient a route is. For example, the
        stress metric helps understand the amount of duplication on a single network link as a result of inefficiencies
        in the overlay routing. Other metrics such as network diameter and graph topology metrics help understand how
        the network would behave in adverse scenarios, such as likelihood of a partition.

    \item {\em Load balancing:} It might be possible to leverage the network layer for implicit load balancing by using a set of nodes as
        representatives for a given content addressable object. This might largely be a function of the service that the
        network layer supports but might be critical to the overall application as well. For instance, a file system can
        use a network layer that can easily balance load for a popular object.

    \item {\em Dynamic membership:} How the routing layer adapts to churn. Nodes are expected to disappear without
        notice in an open network environment. The service layer and the routing layer has to include redundancy in
        order to gracefully cope with such failures. The probability of key nodes disappearing which can lead to network
        partition should be minimized. In our study, we look at the possibility of network partitions as a way to
        understand robustness in design.
    \item {\em Tolerance to Byzantine behavior:} Can the network can tolerate malicious and intentional disruption.
        While correlated failure modeling can provide some approximation to intentional disruptoin, this attack vector
        is something that has not been studied in previous p2p network research. Most overlay protocols are not
        evaluated from this perspective of denial of service attacks, in essence, understanding what type of attack
        would need to be mounted to bring the network down (or cause a partition) and whether such an attack can be
        detected (or mitigated).
\end{itemize}

\subsection{Data collection}

Similar to how traceroute works on the Internet by sending pings with sequentially increasing TTL values, one can create
a similar "route crawling" algorithm that can compute p2p overlay topologies for blockchain networks. For example, with
a prefix-based routing scheme like Kademlia, a crawler can send route requests to a set of carefully determined content
or host IDs which can help it maintain an updated view of the network. This data can be combined in real-time with
mulitple such observers in the network. Note that the physical location (layer 3) of these crawlers has no significance
on the measurment outcome, essentially a single machine can run a few hundred of these crawlers that can be unformly
distributed in the routing space.

We will present results from crawling the top blockchain networks. For some of the upcoming chains, we will also
present results from their respective testnets which we have been provided access to for research purposes. 

\section{Conclusion}
\label{sec:conc}

The overall goal of our research is to cultivate an understanding in the blockchain community for the importance of p2p
overlay network design. We do this in two steps: (i) The first step is to present an empirical study of existing
networks, their performance and understanding of their limitations and bottlenecks if any, (ii) bring forth lessons from
extensive research done in the past decade to the blockchain network design in light of the empirical study. 
This will help researchers and blockchain engineers design a better network backbone to support fully decentralized
applications.

% An example of a floating figure using the graphicx package.
% Note that \label must occur AFTER (or within) \caption.
% For figures, \caption should occur after the \includegraphics.
% Note that IEEEtran v1.7 and later has special internal code that
% is designed to preserve the operation of \label within \caption
% even when the captionsoff option is in effect. However, because
% of issues like this, it may be the safest practice to put all your
% \label just after \caption rather than within \caption{}.
%
% Reminder: the "draftcls" or "draftclsnofoot", not "draft", class
% option should be used if it is desired that the figures are to be
% displayed while in draft mode.
%
%\begin{figure}[!t]
%\centering
%\includegraphics[width=2.5in]{myfigure}
% where an .eps filename suffix will be assumed under latex, 
% and a .pdf suffix will be assumed for pdflatex; or what has been declared
% via \DeclareGraphicsExtensions.
%\caption{Simulation results for the network.}
%\label{fig_sim}
%\end{figure}

% Note that the IEEE typically puts floats only at the top, even when this
% results in a large percentage of a column being occupied by floats.


% An example of a double column floating figure using two subfigures.
% (The subfig.sty package must be loaded for this to work.)
% The subfigure \label commands are set within each subfloat command,
% and the \label for the overall figure must come after \caption.
% \hfil is used as a separator to get equal spacing.
% Watch out that the combined width of all the subfigures on a 
% line do not exceed the text width or a line break will occur.
%
%\begin{figure*}[!t]
%\centering
%\subfloat[Case I]{\includegraphics[width=2.5in]{box}%
%\label{fig_first_case}}
%\hfil
%\subfloat[Case II]{\includegraphics[width=2.5in]{box}%
%\label{fig_second_case}}
%\caption{Simulation results for the network.}
%\label{fig_sim}
%\end{figure*}
%
% Note that often IEEE papers with subfigures do not employ subfigure
% captions (using the optional argument to \subfloat[]), but instead will
% reference/describe all of them (a), (b), etc., within the main caption.
% Be aware that for subfig.sty to generate the (a), (b), etc., subfigure
% labels, the optional argument to \subfloat must be present. If a
% subcaption is not desired, just leave its contents blank,
% e.g., \subfloat[].


% An example of a floating table. Note that, for IEEE style tables, the
% \caption command should come BEFORE the table and, given that table
% captions serve much like titles, are usually capitalized except for words
% such as a, an, and, as, at, but, by, for, in, nor, of, on, or, the, to
% and up, which are usually not capitalized unless they are the first or
% last word of the caption. Table text will default to \footnotesize as
% the IEEE normally uses this smaller font for tables.
% The \label must come after \caption as always.
%
%\begin{table}[!t]
%% increase table row spacing, adjust to taste
%\renewcommand{\arraystretch}{1.3}
% if using array.sty, it might be a good idea to tweak the value of
% \extrarowheight as needed to properly center the text within the cells
%\caption{An Example of a Table}
%\label{table_example}
%\centering
%% Some packages, such as MDW tools, offer better commands for making tables
%% than the plain LaTeX2e tabular which is used here.
%\begin{tabular}{|c||c|}
%\hline
%One & Two\\
%\hline
%Three & Four\\
%\hline
%\end{tabular}
%\end{table}


% Note that the IEEE does not put floats in the very first column
% - or typically anywhere on the first page for that matter. Also,
% in-text middle ("here") positioning is typically not used, but it
% is allowed and encouraged for Computer Society conferences (but
% not Computer Society journals). Most IEEE journals/conferences use
% top floats exclusively. 
% Note that, LaTeX2e, unlike IEEE journals/conferences, places
% footnotes above bottom floats. This can be corrected via the
% \fnbelowfloat command of the stfloats package.


% if have a single appendix:
%\appendix[Proof of the Zonklar Equations]
% or
%\appendix  % for no appendix heading
% do not use \section anymore after \appendix, only \section*
% is possibly needed

% use appendices with more than one appendix
% then use \section to start each appendix
% you must declare a \section before using any
% \subsection or using \label (\appendices by itself
% starts a section numbered zero.)
%

% Can use something like this to put references on a page
% by themselves when using endfloat and the captionsoff option.
%\ifCLASSOPTIONcaptionsoff
%  \newpage
%\fi

% trigger a \newpage just before the given reference
% number - used to balance the columns on the last page
% adjust value as needed - may need to be readjusted if
% the document is modified later
%\IEEEtriggeratref{8}
% The "triggered" command can be changed if desired:
%\IEEEtriggercmd{\enlargethispage{-5in}}

% references section

% can use a bibliography generated by BibTeX as a .bbl file
% BibTeX documentation can be easily obtained at:
% http://mirror.ctan.org/biblio/bibtex/contrib/doc/
% The IEEEtran BibTeX style support page is at:
% http://www.michaelshell.org/tex/ieeetran/bibtex/
\bibliographystyle{IEEEtran}
% \bibliographystyle{unsrt}
% argument is your BibTeX string definitions and bibliography database(s)
%\bibliography{IEEEabrv,../bib/paper}
%
% <OR> manually copy in the resultant .bbl file
% set second argument of \begin to the number of references
% (used to reserve space for the reference number labels box)

\bibliography{pipaper.bib}
% If you have an EPS/PDF photo (graphicx package needed) extra braces are
% needed around the contents of the optional argument to biography to prevent
% the LaTeX parser from getting confused when it sees the complicated
% \includegraphics command within an optional argument. (You could create
% your own custom macro containing the \includegraphics command to make things
% simpler here.)
%\begin{IEEEbiography}[{\includegraphics[width=1in,height=1.25in,clip,keepaspectratio]{mshell}}]{Michael Shell}
% or if you just want to reserve a space for a photo:

% if you will not have a photo at all:
\begin{IEEEbiographynophoto}{Arunesh Mishra:}

Dr. Mishra is an Entrepreneur and a Researcher in the area of Distributed Systems, Networking and
    Security. He has published in top ACM and
    IEEE Conferences including a best paper award at ACM Mobicom. More at https://arunesh.github.io\end{IEEEbiographynophoto}

% insert where needed to balance the two columns on the last page with
% biographies
%\newpage

\begin{IEEEbiographynophoto}{Adi Kancherla:}
    Mr. Kancherla is an ex-Google engineer with a graduate degree from University of Wisconsin, Madison. He has
    experience in building scalable database backends to support high availability, resilience in the presence of
    failures and real-time performance.
\end{IEEEbiographynophoto}

% You can push biographies down or up by placing
% a \vfill before or after them. The appropriate
% use of \vfill depends on what kind of text is
% on the last page and whether or not the columns
% are being equalized.

%\vfill

% Can be used to pull up biographies so that the bottom of the last one
% is flush with the other column.
%\enlargethispage{-5in}



% that's all folks
\end{document}



